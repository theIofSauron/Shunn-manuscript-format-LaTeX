\documentclass[12pt]{article}
\usepackage[margin=1in]{geometry}
\usepackage{fancyhdr}
\usepackage{fontspec}


%The following line only works in XeLaTeX or LuaLaTeX
\setmainfont{Times New Roman}

\makeatletter
\def\@maketitle{
  \let\footnote\thanks
  \vspace*{180pt}
  \begin{center}
    {\normalsize \@title \par}
    {\normalsize by \@author \par}
  \end{center}
  \vspace*{12pt}
}
\makeatother

\title{Proper Manuscript Format}
\author{Bill Shunn}

\pagestyle{fancy}
\fancyhf{}
\fancyhead[R]{Shunn / Format / \thepage}
\renewcommand{\headrulewidth}{0pt}
\renewcommand{\footrulewidth}{0pt}

\begin{document}

\raggedright William Shunn (he/him) 
\hfill
\raggedleft about 1,500 words

\raggedright 12 Courier Place \\
Pica’s Font, NY 12012 \\
(212) 555-1212‬ \\
format@shunn.net\\

\linespread{2}\selectfont

Active member, SCHWA

\setlength\parindent{0.5in}
\setlength{\parskip}{0pt}

\let\originalnewpage\newpage
\let\newpage\relax
\maketitle
\let\newpage\originalnewpage

\thispagestyle{empty}

No one knows how many good stories are passed over because
the manuscripts containing them are poorly formatted. We can be
certain, however, that editors will more eagerly read a cleanly
formatted manuscript than a cluttered and clumsy one. Here are a
few suggestions for giving your manuscript that critical leg up
on the competition.

Start with a fresh white page, no color, no decorations.
Set one-inch margins all around--left, right, top, and bottom.
This is the default for most word processors, but you might want
to recheck your settings just to be safe.

Use black type only, since other colors can make your work
difficult to read. Set your font size to 12 points. For the font itself choose something standard and easily readable, like Times New Roman. Avoid sans-serif fonts, and stay far away from
anything flashy or unusual. You want to wow the editor with your
content, not your font choice. (Some writers, myself included,
still prefer Courier New, a monospaced font that resembles
typewriter output. You can use that too if you like, but it’s
probably on its way out, at least in fiction circles.)

Place your contact information in the upper-left corner of
the first page, including your legal name, address, phone number,
and email. Add your preferred pronouns if you like. List any
professional writing affiliations next, but only when relevant.
If you belong to the Science Fiction and Fantasy Writers of
America, for instance, you should say so on submissions to
\underline{Asimov’s} or \underline{Analog,} but your membership might not cut much ice
with editors at \underline{The New Yorker} or \underline{Cat Fancy.}

In the upper-right corner of the first page, place an
approximate word count. Get this number from your word
processor, then round to the nearest hundred. (This manuscript,
for example, is 1,470 words in length, which rounds to 1,500.)
If you’re edging into novella territory, round to the nearest
500. The point of a word count is not to tell your editor the
exact length of the manuscript, but approximately how much space
your story will take up in the publication.

Though many sources say you should, it is not necessary to
place your Social Security number or any other tax ID on your
manuscript. If your story is accepted, the publisher will ask for it in your contract. Until then, this is extraneous (and in
fact presumptuous) information.

Place the full title of your story a third to halfway down
the first page, centered on its own line. (The editor may use
that empty space to make notes for the production team.) Double-
space once down and center your byline below the title. Your
byline indicates the name that gets credit for the story when
it’s published. This is not necessarily the same as your legal
name up top, which is the one that will be printed (we hope) on
your check. It could be a pen name, or a variation on your legal
name. Even if the two names are identical, each must appear in
its appointed slot.

Double-space two more times down, and that’s where you’ll
start the actual text of your story. As a matter of fact, you
should set your line spacing to double from here forward, because
the full text of your story should be double-spaced. Text reads
more quickly when it has room to breathe, but more importantly
the editor needs room between the lines to mark up your
manuscript with her trusty blue pencil. (This is the case even
with electronic manuscripts, which can be marked up with a stylus
on a tablet screen.)

The first line of every paragraph should be indented one
half-inch from the left margin. Do not place extra line spaces
between paragraphs, as is the common practice in online writing.
First-line indentation is sufficient to indicate that a new
paragraph has begun. (You can set the paragraph formatting in
your word processor to handle indentation for you. This will
also make things easier for the production team when they’re
preparing your story for publication.)

The text of your story should be left-aligned. This means
that, except for paragraph indentations, the left margin of your
manuscript should be ruler-straight, while the right margin
remains ragged. Full justification, in which both margins are
straight, is a typesetting style for finished copy, not for
manuscripts on submission.

Now that we’re moving past the front page, this is a good
time to create the header that should appear on every subsequent
page of your manuscript. This header consists of the surname
from your byline, one or two keywords from the title of your
story, and the page number. It belongs in the upper-right corner
for ready visibility. With your cursor on the second page, open
your word processor’s header/footer feature. Place your header
text flush right, and be sure to specify that the header itself
should not appear on the first page.

That covers most of the high-level aspects of manuscript
formatting. Let’s zoom down to the sentence level now. Standard
practice today is to put only one space between sentences. Back
in the typewriter era, two spaces was the standard, but those
days have flown. For those of you still in the two-space habit,
you might consider doing a quick search-and-replace before
sending off your story, if only to save the production team that
extra step when preparing it for publication.

To emphasize a specific word or phrase in your manuscript,
do so with \textit{italics.} It used to be the practice to \underline{underline} for
emphasis, but that’s because there was no option for italics on
most typewriters. Some publications may still prefer to see
underlining since it stands out a little better on a screen, but
those would be the minority. Consult submission guidelines if
there’s any doubt, and choose italics in the absence of other
instructions.

If you want to indicate an em dash--the punctuation that
sets off this phrase--simply type two hyphens. Most word
processors will convert the two hyphens to a dash automatically.
(Courier users might want to turn off this particular feature of
autocorrect, since in monospaced fonts a dash is difficult to
distinguish from a lone hyphen.) There’s no need to put spaces
around the dash.

“A lot of people ask me about dialog,” I told an editor
friend of mine recently. “Do you have any suggestions?”

“Dialog should be enclosed in quotation marks,” she said.
“Some writers get away with doing it differently, but they’re
rare exceptions.”

“Isn’t it also the usual practice to start a new paragraph
when the speaker changes?” I asked.

“Yes, it is. That helps the reader keep track of who’s
speaking even when speech tags are omitted.”

Speaking of which, you should have the “smart quotes”
feature turned on in your word processor. This converts double
and single quotation marks alike to the appropriate curly
version, either opening or closing, as you type. This too will
be a tremendous help to the production team, but be aware that
smart quotes don’t always work perfectly. Watch especially for
words with leading apostrophes, since autocorrect’ll convert ’em
incorrectly to opening single quotes.

If you want a scene break to appear in your story, center
the symbol “\#” on a line by itself. Don’t just leave the line
blank. As you edit and revise your manuscript prior to
submission, those breaks can shift up or down, and word
processors often hide blank lines that fall at the top or bottom
of a page. You don’t want your editor skipping over your scene
breaks because they accidentally vanished.

Finally, though you don’t need to make any explicit
indication that your story is over, some writers choose to center
the word “END” after the last line of text. This can prevent
ambiguity when your closing words fall near the bottom of the
page.

That’s all there is to it. Now you’re ready to submit that
story! Or are you? This is a good time to read through it at
least once more, checking carefully for typos. One or two errors
won’t earn you an automatic rejection, but you’ll make a better
impression if your first few pages are as clean as possible.

And if you’re planning to mail a hard copy, remember to use
plain white paper and print on only one side of the page.

\begin{center}
\#
\end{center}

While you’ll find some variation in the ways different
writers format their manuscripts, no one departs far from what
I’ve outlined above. Still, you should always check a market’s
submission guidelines before sending your work. If their rules
differ from these, follow theirs.

At the very least, these suggestions will guarantee your
work \underline{looks} professional when it arrives. How the story itself
comes across is an entirely separate matter--and that part’s all
up to you. Knock ’em dead!



\end{document}